% ================================
% INTRODUCTION CHAPTER
% ================================

\section{Introduction}

Large-scale logistics networks such as Amazon's middle-mile system move millions
of packages each day across thousands of facilities, transportation lanes, and
time windows. Constructing feasible and cost-efficient transportation plans
requires solving variants of NP-hard combinatorial optimization problems,
including vehicle routing (VRP), multi-commodity flow (MCF), hub-and-spoke
routing, load rebalancing, and time-window–constrained dispatching. In practice,
these problems are typically formulated as mixed-integer linear programs (MILPs)
and solved repeatedly throughout the day as demand patterns, congestion
conditions, and network availability change. While MILPs offer strong optimality
guarantees, they become computationally prohibitive at the scale of real-world
e-commerce operations, where an instance may contain hundreds of nodes,
thousands of edges, and strict sub-minute latency requirements for dynamic
re-optimization.

Classical heuristics such as greedy consolidation, nearest-hub assignment,
rule-based cycle construction, or local search variants (e.g., LNS) are fast but
fragile. They require extensive hand-tuning, degrade under distribution shift,
and do not incorporate global constraints such as SLA deadlines, congestion
propagation, or multi-hop feasibility. Neural Combinatorial Optimization (Neural
CO) has emerged as a promising alternative, using attention-based models or
graph neural networks to approximate routing heuristics. However, current Neural
CO methods generate solutions in a single shot, lack explicit constraint
handling, and fail to capture the algebraic structure that classical MILPs rely
on—especially binary decisions and coupling constraints. As a result, modern
neural routing models remain significantly weaker than full MILP solvers on
large, structured supply-chain problems.

In this work, we propose a new direction: \textbf{Neural Surrogate MILP
Solvers}—models that do not merely imitate solutions but approximate the
\emph{structure} and \emph{reasoning steps} of MILPs. Our approach, the
\textbf{Neural Surrogate MILP Solver with Constraint Experts}, integrates three
key components:
\begin{enumerate}[leftmargin=1.2em]
    \item \textbf{Constraint Experts:} a Mixture-of-Experts (MoE) module where each
          expert specializes in a different class of MILP constraints (flow conservation,
          capacity coupling, timing/SLA feasibility, binary arc selection).
    \item \textbf{Soft Integer Relaxation:} a differentiable relaxation of binary
          routing decisions, combined with feasibility masks to preserve discrete structure.
    \item \textbf{Latent Gradient Refinement:} a learned refinement operator that
          approximates MILP-style descent, enabling multi-step correction and feasibility
          recovery.
\end{enumerate}

This framework allows deep neural networks to act as amortized, differentiable
approximators to MILP solvers—capturing both high-level routing patterns and
low-level structural constraints, while avoiding the computational overhead of
exact branch-and-bound. We evaluate the proposed method on VRP-style benchmarks
with 50--200 nodes, representing realistic abstractions of Amazon middle-mile
networks. The Neural Surrogate MILP Solver achieves competitive optimality gaps
compared to commercial MILP solvers while providing up to two orders of
magnitude faster inference. Unlike classical heuristics or standard Neural CO
models, our method offers both \emph{speed} and \emph{structured feasibility},
demonstrating its applicability to real-time re-optimization in large supply-chain
systems.

Overall, this work presents a hybrid optimization paradigm that combines the
algebraic structure of MILPs with the flexibility and scalability of deep
learning, opening the door to foundation-style models for industrial-scale
combinatorial optimization.

% ============================================================
\section{Motivation}

Amazon's middle-mile network operates at a scale where thousands of facilities—
Fulfillment Centers (FCs), Sort Centers (SCs), line-haul hubs, and Delivery
Stations (DSs)—must move millions of packages under strict cost, capacity, and
SLA constraints. Each day, the system must repeatedly solve large classes of
NP-hard routing and flow problems, including:
\begin{itemize}[leftmargin=1.3em]
    \item \textbf{VRP/CVRP:} constructing multi-stop transportation sequences under
          vehicle capacities and time windows.
    \item \textbf{Multi-Commodity Flow (MCF):} pushing diverse package volumes through
          constrained lane capacities.
    \item \textbf{Hub-and-Spoke Routing:} selecting consolidation hubs, segmentation
          paths, and inter-hub transfer flows.
    \item \textbf{Load Rebalancing:} shifting excess volume from saturated SCs or HUBs
          toward capacity-rich neighbors.
    \item \textbf{Time-Window/SLA Dispatching:} generating routing plans that meet
          delivery deadlines across regions.
\end{itemize}

These problems are traditionally formulated as large mixed-integer linear
programs (MILPs). While MILPs offer optimality guarantees, they face significant
limitations at Amazon scale:
\begin{itemize}[leftmargin=1.3em]
    \item \textbf{High computational cost:} exact branch-and-bound requires exploring
          millions of integer nodes.
    \item \textbf{Limited real-time responsiveness:} even state-of-the-art solvers
          require minutes for 100--500 node networks.
    \item \textbf{Frequent re-optimization:} demand spikes, congestion changes, weather
          events, and SLA emergencies require re-solving plans dozens of times per hour.
    \item \textbf{Sensitivity to instance structure:} small perturbations in demand or
          lane capacity can dramatically change runtime.
\end{itemize}

Because exact MILPs are expensive, production systems rely on handcrafted
heuristics such as greedy consolidation, nearest-hub dispatching, lane-ranking,
or local-search strategies. However, these heuristics:
\begin{itemize}[leftmargin=1.3em]
    \item fail to incorporate global constraints or downstream congestion,
    \item degrade sharply under distribution shift,
    \item require continuous manual tuning,
    \item produce inconsistent routing behavior across regions.
\end{itemize}

Neural Combinatorial Optimization (Neural CO) attempts to learn routing
heuristics using attention networks, GNNs, or RL. Yet these models still fall
short for operational-scale logistics because they:
\begin{itemize}[leftmargin=1.3em]
    \item generate solutions in a single forward pass,
    \item lack explicit MILP-style constraint enforcement,
    \item have no refinement or feasibility recovery step,
    \item collapse on large, structured VRP/MCF instances.
\end{itemize}

These gaps reveal a fundamental limitation: current neural models fail to
approximate the \emph{reasoning structure} of MILPs. Meanwhile, exact MILPs are
too slow for Amazon's real-time operations. This motivates a new class of
models: \textbf{neural surrogate MILP solvers}—fast, structure-aware,
differentiable approximators capable of real-time repeated optimization across
Amazon's global supply chain.
