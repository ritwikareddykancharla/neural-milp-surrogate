%%%%%%%%%%%%%%%%%%%%%%%%%%%%%%%%%%%%%%%%%%%%%%%%%%%%%%%%%%%%
% NEURIPS 2025 PAPER — Neural Surrogate MILP Solver
%%%%%%%%%%%%%%%%%%%%%%%%%%%%%%%%%%%%%%%%%%%%%%%%%%%%%%%%%%%%

\documentclass{article}
\usepackage[utf8]{inputenc}

% NeurIPS style — change preprint -> final after acceptance
\usepackage[preprint]{neurips_2025}

%%%%%%%%%%%%%%%%%%%%%%%%%%%%%%%%%%%%%%%%%%%%%%%%%%%%%%%%%%%%
% BASIC PACKAGES
%%%%%%%%%%%%%%%%%%%%%%%%%%%%%%%%%%%%%%%%%%%%%%%%%%%%%%%%%%%%

\usepackage{amsmath, amssymb}
\usepackage{graphicx}
\usepackage{booktabs}
\usepackage{enumitem}
\usepackage{url}
\usepackage{subfigure}
\usepackage{algorithm}
\usepackage{algorithmic}
\usepackage[hidelinks]{hyperref}
\usepackage{listings}
\usepackage{xcolor}
\usepackage{tocloft}

\definecolor{codebg}{RGB}{245,245,245}

\lstdefinestyle{pytorch}{
    backgroundcolor=\color{codebg},
    language=Python,
    basicstyle=\ttfamily\small,
    keywordstyle=\color{blue},
    commentstyle=\color{gray},
    stringstyle=\color{orange},
    frame=single,
    rulecolor=\color{black!20},
    showstringspaces=false,
    breaklines=true,
    tabsize=4
}

% Pretty color palette
\definecolor{codebg}{HTML}{FCFCFC}
\definecolor{keyword}{HTML}{3971ED}
\definecolor{comment}{HTML}{6A737D}
\definecolor{string}{HTML}{D14}
\definecolor{purple}{HTML}{A626A4}
\definecolor{number}{HTML}{B3B3B3}

\lstdefinestyle{prettyPython}{
    backgroundcolor=\color{codebg},
    language=Python,
    basicstyle=\ttfamily\footnotesize\color{black},
    keywordstyle=\color{keyword}\bfseries,
    commentstyle=\color{comment}\itshape,
    stringstyle=\color{string},
    numberstyle=\tiny\color{number},
    numbers=left,
    stepnumber=1,
    frame=single,
    framerule=0.3pt,
    rulecolor=\color{black!15},
    tabsize=4,
    showstringspaces=false,
    breaklines=true,
    emph={nn,Parameter,Module,Tensor,softmax,clamp,sigmoid},
    emphstyle=\color{purple},
}

% ===========================================
% Make \section LOOK like a CHAPTER PAGE
% ===========================================
\usepackage{titlesec}

\titleformat{\section}
  {\normalfont\Huge\bfseries}
  {\thesection}
  {1em}
  {}

\titlespacing*{\section}
  {0pt}{4cm}{1cm}

% Add a rule under the chapter title
\newcommand{\sectionrule}{\vspace{0.5cm}\rule{0.7\textwidth}{1pt}\vspace{1cm}}

\let\oldsection\section
\renewcommand{\section}[1]{
    \clearpage
    \oldsection{#1}
    \sectionrule
}

\titleformat{\subsection}
  {\large\bfseries}
  {\thesubsection}
  {1em}
  {}


\usepackage[none]{hyphenat}
\sloppy

\setlength{\parindent}{1.5em}   % paragraph indent
\setlength{\parskip}{0.7em}     % space BETWEEN paragraphs

%%%%%%%%%%%%%%%%%%%%%%%%%%%%%%%%%%%%%%%%%%%%%%%%%%%%%%%%%%%%
% COLORS & AESTHETICS
%%%%%%%%%%%%%%%%%%%%%%%%%%%%%%%%%%%%%%%%%%%%%%%%%%%%%%%%%%%%

\usepackage{xcolor}

% Soft NeurIPS color theme
\definecolor{softblue}{RGB}{88, 140, 205}
\definecolor{softbluefill}{RGB}{230, 240, 255}
\definecolor{softgray}{RGB}{245, 245, 245}

%%%%%%%%%%%%%%%%%%%%%%%%%%%%%%%%%%%%%%%%%%%%%%%%%%%%%%%%%%%%
% TIKZ FOR DIAGRAMS
%%%%%%%%%%%%%%%%%%%%%%%%%%%%%%%%%%%%%%%%%%%%%%%%%%%%%%%%%%%%

\usepackage{tikz}
\usetikzlibrary{positioning, arrows.meta, shapes, fit}

% TikZ styles for pretty neural diagrams
\tikzset{
    neuronbox/.style={
        draw=softblue,
        fill=softbluefill,
        thick,
        rounded corners,
        minimum width=3.6cm,
        minimum height=1.0cm,
        align=center
    },
    modulebox/.style={
        draw=softblue,
        fill=softbluefill,
        very thick,
        rounded corners=4pt,
        minimum width=4.8cm,
        minimum height=1.2cm,
        align=center
    },
    arrow/.style={
        -{Triangle[length=3mm,width=4mm]},
        thick,
        softblue
    }
}

\setlength{\cftbeforesecskip}{6pt}
\setlength{\cftaftertoctitleskip}{12pt}

\hypersetup{
    colorlinks=true,
    linkcolor=blue,
    urlcolor=blue,
    citecolor=purple
}

\usepackage{titlesec}

% Make abstract heading huge like a chapter
\titleformat{\abstractname}
  {\normalfont\Huge\bfseries}{ }{0pt}{}

% Add extra spacing
\addto\captionsenglish{\renewcommand{\abstractname}{Abstract}}

\makeatletter
\renewenvironment{abstract}{
    \cleardoublepage
    \begin{center}
        {\Huge\bfseries Abstract}\\[0.5cm]
        \rule{0.7\textwidth}{1pt}\\[1cm]
    \end{center}
    \begin{quotation}
}{
    \end{quotation}
}
\makeatother


%%%%%%%%%%%%%%%%%%%%%%%%%%%%%%%%%%%%%%%%%%%%%%%%%%%%%%%%%%%%
% OPTIONAL: NICER ALGO CAPTIONS
%%%%%%%%%%%%%%%%%%%%%%%%%%%%%%%%%%%%%%%%%%%%%%%%%%%%%%%%%%%%
\renewcommand{\algorithmiccomment}[1]{\hfill \textcolor{gray}{\# #1}}

%%%%%%%%%%%%%%%%%%%%%%%%%%%%%%%%%%%%%%%%%%%%%%%%%%%%%%%%%%%%
% TITLE & AUTHOR
%%%%%%%%%%%%%%%%%%%%%%%%%%%%%%%%%%%%%%%%%%%%%%%%%%%%%%%%%%%%

\title{Neural MILP Surrogate Solver with Constraint Experts}

\author{
  Ritwika Kancharla \\
  \texttt{ritwikareddykancharla@gmail.com}
}

%%%%%%%%%%%%%%%%%%%%%%%%%%%%%%%%%%%%%%%%%%%%%%%%%%%%%%%%%%%%
\begin{document}
\maketitle
\tableofcontents
\clearpage

%%%%%%%%%%%%%%%%%%%%%%%%%%%%%%%%%%%%%%%%%%%%%%%%%%%%%%%%%%%%
% ABSTRACT
%%%%%%%%%%%%%%%%%%%%%%%%%%%%%%%%%%%%%%%%%%%%%%%%%%%%%%%%%%%%

\begin{abstract}
Large-scale routing and supply-chain systems—such as those used in Amazon’s middle-mile network—rely heavily on mixed-integer linear programs (MILPs) to enforce capacity, flow conservation, and service-level constraints across thousands of daily transportation decisions. However, classical MILP solvers struggle to meet the strict latency and throughput requirements of operational routing, where even moderately sized vehicle-routing problems (VRPs) can take minutes or hours to solve and must be re-computed whenever demand or network conditions shift. We propose a \textbf{Neural Surrogate MILP Solver}, a deep learning model that approximates MILP structure using a differentiable architecture designed to mimic core solver behaviors. Our method combines (i) a \textit{Constraint Expert Mixture} that learns the roles of different MILP constraint families, (ii) a \textit{soft integer relaxation} that provides a neural analogue of discrete decision boundaries, and (iii) a \textit{latent gradient refinement} procedure that performs AGI-style iterative optimization in solution space. Together, these components enable a fast, amortized neural solver that produces high-quality solutions for 50--200 node VRP instances at up to 10–100$\times$ lower latency than classical MILP solvers while maintaining competitive optimality gaps. Our results highlight the feasibility of building neural, general-purpose surrogate optimizers capable of approximating MILPs at Amazon-scale, opening a path toward hybrid AGI-inspired planning systems for real-world logistics and transportation networks.
\end{abstract}


%%%%%%%%%%%%%%%%%%%%%%%%%%%%%%%%%%%%%%%%%%%%%%%%%%%%%%%%%%%%
% 1. INTRODUCTION
%%%%%%%%%%%%%%%%%%%%%%%%%%%%%%%%%%%%%%%%%%%%%%%%%%%%%%%%%%%%
% \chapter{Introduction}
% ================================
% INTRODUCTION CHAPTER
% ================================

\section{Introduction}

Large-scale logistics networks such as Amazon's middle-mile system move millions
of packages each day across thousands of facilities, transportation lanes, and
time windows. Constructing feasible and cost-efficient transportation plans
requires solving variants of NP-hard combinatorial optimization problems,
including vehicle routing (VRP), multi-commodity flow (MCF), hub-and-spoke
routing, load rebalancing, and time-window–constrained dispatching. In practice,
these problems are typically formulated as mixed-integer linear programs (MILPs)
and solved repeatedly throughout the day as demand patterns, congestion
conditions, and network availability change. While MILPs offer strong optimality
guarantees, they become computationally prohibitive at the scale of real-world
e-commerce operations, where an instance may contain hundreds of nodes,
thousands of edges, and strict sub-minute latency requirements for dynamic
re-optimization.

Classical heuristics such as greedy consolidation, nearest-hub assignment,
rule-based cycle construction, or local search variants (e.g., LNS) are fast but
fragile. They require extensive hand-tuning, degrade under distribution shift,
and do not incorporate global constraints such as SLA deadlines, congestion
propagation, or multi-hop feasibility. Neural Combinatorial Optimization (Neural
CO) has emerged as a promising alternative, using attention-based models or
graph neural networks to approximate routing heuristics. However, current Neural
CO methods generate solutions in a single shot, lack explicit constraint
handling, and fail to capture the algebraic structure that classical MILPs rely
on—especially binary decisions and coupling constraints. As a result, modern
neural routing models remain significantly weaker than full MILP solvers on
large, structured supply-chain problems.

In this work, we propose a new direction: \textbf{Neural Surrogate MILP
Solvers}—models that do not merely imitate solutions but approximate the
\emph{structure} and \emph{reasoning steps} of MILPs. Our approach, the
\textbf{Neural Surrogate MILP Solver with Constraint Experts}, integrates three
key components:
\begin{enumerate}[leftmargin=1.2em]
    \item \textbf{Constraint Experts:} a Mixture-of-Experts (MoE) module where each
          expert specializes in a different class of MILP constraints (flow conservation,
          capacity coupling, timing/SLA feasibility, binary arc selection).
    \item \textbf{Soft Integer Relaxation:} a differentiable relaxation of binary
          routing decisions, combined with feasibility masks to preserve discrete structure.
    \item \textbf{Latent Gradient Refinement:} a learned refinement operator that
          approximates MILP-style descent, enabling multi-step correction and feasibility
          recovery.
\end{enumerate}

This framework allows deep neural networks to act as amortized, differentiable
approximators to MILP solvers—capturing both high-level routing patterns and
low-level structural constraints, while avoiding the computational overhead of
exact branch-and-bound. We evaluate the proposed method on VRP-style benchmarks
with 50--200 nodes, representing realistic abstractions of Amazon middle-mile
networks. The Neural Surrogate MILP Solver achieves competitive optimality gaps
compared to commercial MILP solvers while providing up to two orders of
magnitude faster inference. Unlike classical heuristics or standard Neural CO
models, our method offers both \emph{speed} and \emph{structured feasibility},
demonstrating its applicability to real-time re-optimization in large supply-chain
systems.

Overall, this work presents a hybrid optimization paradigm that combines the
algebraic structure of MILPs with the flexibility and scalability of deep
learning, opening the door to foundation-style models for industrial-scale
combinatorial optimization.

% ============================================================
\section{Motivation}

Amazon's middle-mile network operates at a scale where thousands of facilities—
Fulfillment Centers (FCs), Sort Centers (SCs), line-haul hubs, and Delivery
Stations (DSs)—must move millions of packages under strict cost, capacity, and
SLA constraints. Each day, the system must repeatedly solve large classes of
NP-hard routing and flow problems, including:
\begin{itemize}[leftmargin=1.3em]
    \item \textbf{VRP/CVRP:} constructing multi-stop transportation sequences under
          vehicle capacities and time windows.
    \item \textbf{Multi-Commodity Flow (MCF):} pushing diverse package volumes through
          constrained lane capacities.
    \item \textbf{Hub-and-Spoke Routing:} selecting consolidation hubs, segmentation
          paths, and inter-hub transfer flows.
    \item \textbf{Load Rebalancing:} shifting excess volume from saturated SCs or HUBs
          toward capacity-rich neighbors.
    \item \textbf{Time-Window/SLA Dispatching:} generating routing plans that meet
          delivery deadlines across regions.
\end{itemize}

These problems are traditionally formulated as large mixed-integer linear
programs (MILPs). While MILPs offer optimality guarantees, they face significant
limitations at Amazon scale:
\begin{itemize}[leftmargin=1.3em]
    \item \textbf{High computational cost:} exact branch-and-bound requires exploring
          millions of integer nodes.
    \item \textbf{Limited real-time responsiveness:} even state-of-the-art solvers
          require minutes for 100--500 node networks.
    \item \textbf{Frequent re-optimization:} demand spikes, congestion changes, weather
          events, and SLA emergencies require re-solving plans dozens of times per hour.
    \item \textbf{Sensitivity to instance structure:} small perturbations in demand or
          lane capacity can dramatically change runtime.
\end{itemize}

Because exact MILPs are expensive, production systems rely on handcrafted
heuristics such as greedy consolidation, nearest-hub dispatching, lane-ranking,
or local-search strategies. However, these heuristics:
\begin{itemize}[leftmargin=1.3em]
    \item fail to incorporate global constraints or downstream congestion,
    \item degrade sharply under distribution shift,
    \item require continuous manual tuning,
    \item produce inconsistent routing behavior across regions.
\end{itemize}

Neural Combinatorial Optimization (Neural CO) attempts to learn routing
heuristics using attention networks, GNNs, or RL. Yet these models still fall
short for operational-scale logistics because they:
\begin{itemize}[leftmargin=1.3em]
    \item generate solutions in a single forward pass,
    \item lack explicit MILP-style constraint enforcement,
    \item have no refinement or feasibility recovery step,
    \item collapse on large, structured VRP/MCF instances.
\end{itemize}

These gaps reveal a fundamental limitation: current neural models fail to
approximate the \emph{reasoning structure} of MILPs. Meanwhile, exact MILPs are
too slow for Amazon's real-time operations. This motivates a new class of
models: \textbf{neural surrogate MILP solvers}—fast, structure-aware,
differentiable approximators capable of real-time repeated optimization across
Amazon's global supply chain.



\section{Background}

This section reviews the core optimization frameworks used in large-scale routing and the emerging class of neural methods that aim to approximate them.

\subsection{Mixed-Integer Linear Programming (MILP)}

Mixed-Integer Linear Programs are a standard tool for modeling routing, assignment, scheduling, and flow problems in logistics. A MILP optimizes a linear objective subject to linear constraints, with a subset of decision variables constrained to be binary or integer. These discrete variables represent structural decisions such as:
\begin{itemize}[leftmargin=1.3em]
    \item selecting transportation lanes,
    \item activating vehicle routes,
    \item choosing consolidation or hub assignments,
    \item enforcing time-window feasibility.
\end{itemize}

MILPs provide strong optimality guarantees but are NP-hard in general. Solvers such as Gurobi and CPLEX rely on branch-and-bound, cutting planes, and presolve heuristics, which can be computationally expensive. For logistics-scale problems (hundreds of nodes and thousands of edges), solving a MILP repeatedly throughout the day becomes infeasible.

\subsection{Classical Routing and Flow Problems}

Amazon-style transportation planning commonly relies on several canonical formulations:
\begin{itemize}[leftmargin=1.3em]
    \item \textbf{Vehicle Routing Problem (VRP):} constructing capacity-constrained routes for trucks and line-haul vehicles.
    \item \textbf{Capacitated VRP (CVRP):} general VRP with volume limits per vehicle.
    \item \textbf{Multi-Commodity Flow (MCF):} pushing heterogeneous package flows across a capacitated directed graph.
    \item \textbf{Time-Window Routing:} ensuring delivery deadlines and SLA thresholds.
    \item \textbf{Hub-and-Spoke Routing:} selecting intermediate consolidation or routing centers.
\end{itemize}

These formulations capture the underlying logistics structure but are computationally prohibitive at the scale required by real-world operations.

\subsection{Heuristics and Metaheuristics}

To address the runtime limitations of MILPs, practitioners often employ heuristic approaches such as:
\begin{itemize}[leftmargin=1.3em]
    \item greedy or nearest-neighbor routing,
    \item hub-ranking and lane-scoring heuristics,
    \item local search methods (e.g., 2-opt, 3-opt, Large Neighborhood Search),
    \item rule-based or simulation-driven policies.
\end{itemize}

While fast, these techniques require manual tuning and often fail to adapt across geographic regions, peak-season modes, or sudden demand shifts. They also provide no guarantees of global feasibility or cost efficiency.

\subsection{Neural Combinatorial Optimization}

Neural Combinatorial Optimization (Neural CO) seeks to learn routing heuristics using neural architectures. Early methods such as Pointer Networks framed routing as sequence generation, while later methods used graph neural networks (GNNs), attention mechanisms, or reinforcement learning to model route construction.

Despite progress, existing Neural CO models have several limitations:
\begin{itemize}[leftmargin=1.3em]
    \item They produce \emph{single-shot} solutions without iterative refinement.
    \item They lack explicit constraint handling for flow conservation, lane capacity, and timing feasibility.
    \item They focus on small-scale VRP (20--100 nodes) and struggle to generalize to industrial-scale networks.
    \item They do not approximate the underlying algebraic structure enforced by MILPs.
\end{itemize}

These limitations prevent existing Neural CO models from serving as reliable replacements for MILPs in real-world routing systems.

\subsection{Toward Neural Surrogates for Optimization}

Recent advances in deep learning---including differentiable optimization layers, neural relaxations, and Mixture-of-Experts (MoE) architectures---suggest the possibility of models that approximate optimization processes themselves. Instead of learning end-to-end mappings from problem instance to solution, such models attempt to learn \emph{structure-preserving approximations} of classical solvers.

However, applying these techniques to industrial-scale routing remains largely unexplored. This gap motivates the approach developed in this work: a neural surrogate model that captures the constraint coupling, discrete structure, and iterative refinement behavior characteristic of MILPs, while offering the speed of modern deep networks.

%%%%%%%%%%%%%%%%%%%%%%%%%%%%%%%%%%%%%%%%%%%%%%%%%%%%%%%%%%%%
% 2. BACKGROUND
%%%%%%%%%%%%%%%%%%%%%%%%%%%%%%%%%%%%%%%%%%%%%%%%%%%%%%%%%%%%


\section{Background}

\subsection{Mixed-Integer Linear Programs}
% Formal MILP definition, constraints, variables.

\subsection{Neural Combinatorial Optimization}
% Pointer nets, transformers, GNNs, neural heuristics.

\subsection{Differentiable Optimization Layers}
% OptNet, deep implicit layers, relaxation techniques.

%%%%%%%%%%%%%%%%%%%%%%%%%%%%%%%%%%%%%%%%%%%%%%%%%%%%%%%%%%%%
% 3. PROBLEM FORMULATION
%%%%%%%%%%%%%%%%%%%%%%%%%%%%%%%%%%%%%%%%%%%%%%%%%%%%%%%%%%%%

\section{Problem Formulation}
\label{sec:problem}

We consider a generic middle-mile routing and flow-allocation problem representative of Amazon-scale logistics. Packages must be moved from upstream facilities (e.g., FCs) to downstream facilities (e.g., DSs) through an intermediate network of Sort Centers (SCs) and hubs, while respecting lane capacities, facility limits, and delivery deadlines.

\subsection{Network and Demand Model}

We model the transportation system as a directed graph
\[
G = (V, E),
\]
where
\begin{itemize}[leftmargin=1.3em]
    \item $V$ is the set of facilities (FC, SC, HUB, DS),
    \item $E \subseteq V \times V$ is the set of transportation lanes.
\end{itemize}

Each lane $(i,j) \in E$ has:
\begin{itemize}[leftmargin=1.3em]
    \item transportation cost $c_{ij} \ge 0$,
    \item travel time $\Delta_{ij} \ge 0$,
    \item capacity $U_{ij} \ge 0$.
\end{itemize}

We assume a set of \emph{demands} (commodities) indexed by $k \in \mathcal{K}$, where each commodity $k$ corresponds to a group of packages that must move from origin $o_k \in V$ to destination $d_k \in V$ with volume $q_k > 0$ and desired arrival deadline $\tau_k$.

\subsection{Decision Variables}

The MILP uses the following decision variables:
\begin{itemize}[leftmargin=1.3em]
    \item $y_{ij} \in \{0,1\}$: lane activation variable, equal to $1$ if lane $(i,j)$ is used by any flow.
    \item $f_{ij}^k \ge 0$: flow of commodity $k$ routed along lane $(i,j)$.
    \item $t_i^k \ge 0$: arrival time of commodity $k$ at node $i$ (for time-window and SLA modeling).
    \item $z_k \ge 0$: lateness slack for commodity $k$ at its destination.
\end{itemize}

Intuitively, $y$ captures the discrete network structure (which lanes are “on”), $f$ describes how package volume moves through the active network, and $t,z$ capture temporal feasibility and SLA violations.

\subsection{MILP Objective}

A typical middle-mile objective balances transportation cost and SLA penalties. We define:
\begin{equation}
\label{eq:objective}
\min_{y, f, t, z} 
\quad 
\underbrace{\sum_{k \in \mathcal{K}} \sum_{(i,j) \in E} c_{ij} f_{ij}^k}_{\text{transportation cost}}
\;+\;
\underbrace{\lambda_{\text{late}} \sum_{k \in \mathcal{K}} z_k}_{\text{SLA penalties}},
\end{equation}
where $\lambda_{\text{late}} > 0$ controls the trade-off between cost and lateness.

\subsection{Flow Conservation Constraints (Flow Expert)}

For each commodity $k \in \mathcal{K}$ and each node $i \in V$, we enforce flow conservation:
\begin{equation}
\label{eq:flow_conservation}
\sum_{j : (i,j) \in E} f_{ij}^k - \sum_{j : (j,i) \in E} f_{ji}^k
=
\begin{cases}
+ q_k, & i = o_k, \\
- q_k, & i = d_k, \\
0,     & \text{otherwise.}
\end{cases}
\end{equation}
These constraints ensure that each commodity’s total outgoing flow from its origin equals its demand, all volume is absorbed at the destination, and intermediate facilities act as pure transshipment nodes.

\subsection{Capacity Constraints (Capacity Expert)}

Lane and node capacities limit the amount of volume that can traverse the network:
\begin{align}
\sum_{k \in \mathcal{K}} f_{ij}^k 
&\le U_{ij} y_{ij}, && \forall (i,j) \in E,
\label{eq:lane_capacity}
\\[2mm]
\sum_{k \in \mathcal{K}} \sum_{j : (i,j) \in E} f_{ij}^k 
&\le C_i, && \forall i \in V,
\label{eq:node_capacity}
\end{align}
where $C_i$ denotes the processing capacity of facility $i$.  
Constraint~\eqref{eq:lane_capacity} couples continuous flows with binary activation variables, making the problem mixed-integer and nontrivial to solve at scale.

\subsection{Routing Structure Constraints (Binary Expert)}

The discrete structure of the active network is encoded through:
\begin{equation}
\label{eq:binary_domain}
y_{ij} \in \{0,1\}, \qquad \forall (i,j) \in E.
\end{equation}

In classical VRP formulations, $y$ may also represent vehicle-level routing decisions (e.g., whether a truck traverses $(i,j)$), often accompanied by additional subtour-elimination or path-connectivity constraints. In our flow-based abstraction, $y$ primarily controls lane activation, but the same binary structure appears in VRP MILPs.

\subsection{Timing and SLA Constraints (Temporal Expert)}

We model travel times and deadlines using a big-$M$ formulation. For each commodity $k$ and each lane $(i,j) \in E$:
\begin{equation}
\label{eq:time_propagation}
t_j^k \;\ge\; t_i^k + \Delta_{ij} - M (1 - y_{ij}),
\end{equation}
where $M$ is a sufficiently large constant. This enforces temporal consistency along active lanes.

At the destination node $d_k$, we define lateness:
\begin{align}
t_{d_k}^k &\le \tau_k + z_k, \label{eq:deadline_relax} \\
z_k &\ge 0. \label{eq:lateness_nonneg}
\end{align}
If a route for commodity $k$ arrives after the deadline $\tau_k$, the slack $z_k$ becomes positive and is penalized in the objective~\eqref{eq:objective}.

\subsection{Complete MILP}

Collecting the components above, the full MILP reads:
\begin{align}
\min_{y, f, t, z} \quad 
& \sum_{k \in \mathcal{K}} \sum_{(i,j) \in E} c_{ij} f_{ij}^k
\;+\; \lambda_{\text{late}} \sum_{k \in \mathcal{K}} z_k
\tag{MILP}
\\[2mm]
\text{s.t.} \quad
& \text{Flow conservation \eqref{eq:flow_conservation}} \nonumber \\
& \text{Capacity constraints \eqref{eq:lane_capacity}--\eqref{eq:node_capacity}} \nonumber \\
& \text{Timing and SLA constraints \eqref{eq:time_propagation}--\eqref{eq:lateness_nonneg}} \nonumber \\
& y_{ij} \in \{0,1\}, \quad f_{ij}^k \ge 0, \quad t_i^k \ge 0, \quad z_k \ge 0. \nonumber
\end{align}

\subsection{Constraint Families and Neural Surrogate View}

The MILP above decomposes naturally into \emph{constraint families}:
\begin{itemize}[leftmargin=1.3em]
    \item \textbf{Flow constraints} \eqref{eq:flow_conservation},
    \item \textbf{Capacity constraints} \eqref{eq:lane_capacity}--\eqref{eq:node_capacity},
    \item \textbf{Binary structure constraints} \eqref{eq:binary_domain},
    \item \textbf{Temporal and SLA constraints} \eqref{eq:time_propagation}--\eqref{eq:lateness_nonneg}.
\end{itemize}

Classical MILP solvers enforce these constraints via branch-and-bound, cutting planes, and local search. In the next section, we describe how our Neural Surrogate MILP Solver mirrors this decomposition using a set of learned \emph{constraint experts}, a soft integer relaxation of $y$, and a latent gradient refinement procedure that mimics solver-like iterative improvement while remaining fully differentiable.

\subsection{How MILPs Are Solved in Practice (Matrix Form and Algorithms)}

In practice, large-scale MILPs of the form described above are expressed in compact matrix notation:
\begin{align}
\min_{x} \quad & c^\top x \\
\text{s.t.} \quad 
& A x = b, \\
& G x \le h, \\
& x_i \in \{0,1\} \;\text{ for } i \in \mathcal{B}, \qquad
x_j \ge 0 \;\text{ for } j \notin \mathcal{B},
\end{align}
where:
\begin{itemize}[leftmargin=1.3em]
    \item $x$ concatenates all decision variables $y, f, t, z$,
    \item $A$ encodes flow-conservation constraints,
    \item $G$ encodes capacity, timing, and feasibility constraints,
    \item $c$ contains lane costs and lateness penalties,
    \item $\mathcal{B}$ indexes the binary variables.
\end{itemize}

For example, the flow-conservation constraints in \eqref{eq:flow_conservation} appear in $A x = b$ as a sparse node–arc incidence matrix, while lane capacity constraints \eqref{eq:lane_capacity} appear as rows in $Gx \le h$ coupling flows and binary variables.

\paragraph{LP Relaxation.}
Commercial solvers begin by solving the linear programming (LP) relaxation of the MILP:
\[
x_i \in [0,1] \quad \text{for } i \in \mathcal{B},
\]
which can be expressed compactly as:
\[
\min_x c^\top x \quad \text{s.t. } A x = b,\; G x \le h.
\]
LP relaxations are solved with variants of the simplex method or interior-point methods.  
This produces a fractional solution $\hat{x}$ that guides further search.

\paragraph{Branch-and-Bound.}
MILPs are solved using \emph{branch-and-bound}, which recursively partitions the feasible region:
\begin{itemize}[leftmargin=1.3em]
    \item If $\hat{x}$ is integer feasible, it is a candidate optimum.
    \item If not, the solver selects a fractional binary variable $x_i$ and branches:
    \[
    x_i = 0 \quad \text{or} \quad x_i = 1.
    \]
\end{itemize}
Each branch solves a new LP relaxation, and the search tree is pruned using lower bounds on the objective.

\paragraph{Cutting Planes.}
Modern solvers add cutting planes (valid inequalities) of the form:
\[
\alpha^\top x \le \beta,
\]
derived from the polyhedral structure of routing and flow constraints.  
Typical cuts include:
\begin{itemize}[leftmargin=1.3em]
    \item capacity cuts,
    \item flow-cover inequalities,
    \item subtour elimination constraints,
    \item knapsack cover cuts.
\end{itemize}

Cuts tighten the LP relaxation and reduce the need for deep branching.

\paragraph{Presolve + Matrix Compression.}
Before solving, solvers reorder rows/columns of $A$ and $G$ to reduce fill-in, remove redundant constraints, and eliminate impossible arcs.  
For Amazon-scale graphs, the matrices involved are sparse with millions of nonzero entries; efficient sparse-matrix operations are crucial.

\paragraph{Practical Limitations at Amazon Scale.}
Even with matrix compression, LP warm starts, and state-of-the-art cuts:
\begin{itemize}[leftmargin=1.3em]
    \item the branch-and-bound tree may contain millions of nodes,
    \item slight demand changes create different fractional patterns $\hat{x}$,
    \item LP solves dominate runtime (interior-point is cubic in worst case),
    \item real-time re-optimization (sub-minute) is infeasible for 100–500 node problems.
\end{itemize}

Thus, matrix-based classical solvers are robust but fundamentally too computationally heavy for high-frequency, Amazon-scale routing workloads.  
This motivates amortized, neural surrogate approaches that learn to approximate the MILP solution mapping without explicitly solving large LPs or exploring deep branch-and-bound trees.

\subsection{From Classical MILPs to Neural Surrogates}

The formulation above reveals two key properties of large-scale routing MILPs:  
(1) their structure is highly regular, with constraints grouped into a small number of repeating families (flow, capacity, binary structure, temporal feasibility), and  
(2) their computational bottleneck lies in repeatedly solving large sparse LP relaxations and traversing deep branch-and-bound trees.

These observations suggest that much of the solver’s reasoning process is \emph{structurally predictable}.  
Instead of recomputing LP relaxations from scratch, a model could learn:
\begin{itemize}[leftmargin=1.3em]
    \item how constraint families interact,
    \item where violations tend to occur,
    \item which arcs are likely to be active in near-optimal solutions,
    \item and how to refine fractional or infeasible assignments into feasible ones.
\end{itemize}

Transformer architectures---especially those augmented with Mixture-of-Experts (MoE) layers---are well-suited for this setting: they can attend over large graphs, route information through specialized constraint experts, and perform iterative refinement akin to solver-style descent. This motivates our Neural Surrogate MILP Solver, which approximates MILP reasoning patterns using differentiable components instead of explicit branch-and-bound.

% Define your routing MILP here.
% Variables, constraints, objective.
% Explain where cost, capacity, flow conservation, etc. come from.


\section{Neural Surrogate MILP Solver}
\label{sec:method}

Classical MILPs enforce feasibility through algebraic constraints and solve for optimality through LP relaxations and branch-and-bound. Our goal is to build a \emph{neural surrogate} that preserves the structure of MILP reasoning while enabling fast, amortized inference. The proposed model consists of three components:
\begin{enumerate}[leftmargin=1.3em]
    \item a transformer encoder that embeds the logistics graph and MILP coefficients,
    \item a Mixture-of-Experts (MoE) module that specializes in different constraint families,
    \item a latent refinement step that approximates MILP-style descent and feasibility recovery.
\end{enumerate}

The architecture takes a MILP instance $(A, G, b, h, c)$ along with graph features as input and outputs approximate lane activations $\hat{y}$, flows $\hat{f}$, and arrival times $\hat{t}$.

\subsection{Input Representation}

Each transportation lane $(i,j) \in E$ is embedded using a feature vector
\[
e_{ij} = \mathrm{Embed}(c_{ij}, U_{ij}, \Delta_{ij}, \text{facility\_types}(i,j)).
\]
Each node $i \in V$ is embedded using
\[
v_i = \mathrm{Embed}(C_i, \text{facility\_type}(i), \text{degree}(i)).
\]

MILP coefficients are also embedded:
\begin{itemize}[leftmargin=1.3em]
    \item rows of $A$ (flow constraints),
    \item rows of $G$ (capacity + time),
    \item objective coefficients $c$,
    \item binary indices $\mathcal{B}$.
\end{itemize}

We form a unified token sequence:
\[
X = [\; v_i\;|\; e_{ij}\;|\; A_r\;|\; G_s\;|\; c\; ],
\]
where $A_r$ and $G_s$ are row embeddings.

\subsection{Transformer Encoder}

A multi-layer transformer encoder processes $X$:
\[
H = \mathrm{TransformerEncoder}(X).
\]

Self-attention allows constraints to attend to variables, variables to attend to graph structure, and cost coefficients to influence routing behavior. This forms a learned analogue of the LP relaxation’s KKT coupling.

\subsection{Constraint Expert Mixture}

MILP constraints fall naturally into families. We introduce four experts:

\[
\text{FlowExpert},\quad
\text{CapacityExpert},\quad
\text{BinaryExpert},\quad
\text{TemporalExpert}.
\]

Each expert $E_k$ learns an update rule of the form:
\[
u^{(k)} = E_k(H),
\]
where $u^{(k)}$ is a correction direction for variables that violate the corresponding constraint family.

A gating network selects the top-$2$ experts per token:
\[
\alpha_k = \mathrm{softmax}(W_g H), \qquad
u = \sum_{k \in \text{Top-2}(\alpha)} \alpha_k\, u^{(k)}.
\]

This mimics the branching heuristics of MILP solvers, where specific constraint violations trigger specialized reasoning.

\subsection{Soft Integer Relaxation}

Binary MILP variables $y_{ij} \in \{0,1\}$ are approximated using a differentiable relaxation:
\[
\tilde{y}_{ij} = \sigma(\ell_{ij}),
\]
where $\ell_{ij}$ are logits predicted from $H$.

To push relaxed solutions toward true binary structure, we apply a projected update:
\[
\hat{y}_{ij} = \mathrm{clip}\big(\tilde{y}_{ij} + \eta \, u_{ij}^{(\text{Binary})},\, 0, 1\big),
\]
where $u_{ij}^{(\text{Binary})}$ is the update from the BinaryExpert and $\eta$ is a learned step size.

This corresponds to a neural analogue of a branch-and-bound fractional tightening step.

\subsection{Flow and Capacity Projection}

Flow variables are updated via:
\[
\tilde{f} = \mathrm{ReLU}(W_f H),
\]
followed by a projection enforcing capacity feasibility:
\[
\hat{f}_{ij}^k = 
\min\left\{
\tilde{f}_{ij}^k, \;
U_{ij} \hat{y}_{ij}
\right\}.
\]

Flow conservation is encouraged through an expert-driven correction:
\[
\hat{f} \leftarrow \hat{f} - \gamma \, u^{(\text{Flow})},
\]
where $u^{(\text{Flow})}$ pushes flows toward satisfying Eq.~\eqref{eq:flow_conservation}.

\subsection{Temporal Refinement}

Arrival times are predicted via:
\[
\tilde{t}_i = W_t H_i,
\]
and then refined using TemporalExpert updates, approximating feasibility of:
\[
t_j \ge t_i + \Delta_{ij} \quad \text{if } \hat{y}_{ij}=1.
\]

\subsection{Latent Gradient Refinement}

After an initial forward pass, we apply $T$ learned refinement iterations:
\[
H^{(t+1)} = H^{(t)} + \rho \cdot \Phi\big(H^{(t)}, u^{(t)}\big),
\]
where $\Phi$ is an MLP that simulates a solver-style descent step.

This mimics:
\begin{itemize}[leftmargin=1.3em]
    \item LP re-solves,
    \item cutting-plane adjustments,
    \item feasibility restoration,
    \item dual descent behavior.
\end{itemize}

The refinement loop significantly improves feasibility and reduces optimality gaps.

\subsection{Training Objective}

The model is trained end-to-end using:
\[
\mathcal{L}
=
\lambda_{\text{cost}} \, \mathcal{L}_{\text{cost}}
+ \lambda_{\text{flow}} \, \mathcal{L}_{\text{flow}}
+ \lambda_{\text{cap}} \mathcal{L}_{\text{cap}}
+ \lambda_{\text{bin}} \mathcal{L}_{\text{binary}}
+ \lambda_{\text{temp}} \mathcal{L}_{\text{time}}.
\]

Where:
\begin{itemize}[leftmargin=1.3em]
    \item $\mathcal{L}_{\text{cost}}$ penalizes deviation from MILP objective,
    \item $\mathcal{L}_{\text{flow}}$ penalizes flow-conservation violation,
    \item $\mathcal{L}_{\text{cap}}$ penalizes capacity violations,
    \item $\mathcal{L}_{\text{binary}}$ encourages $\hat{y}$ toward $\{0,1\}$,
    \item $\mathcal{L}_{\text{time}}$ penalizes SLA or time-window violations.
\end{itemize}

This mirrors the structure of MILP feasibility and optimality conditions, allowing the model to learn solver-like reasoning.

\subsection{Inference}

At inference time, the model performs a single forward pass plus a small number of refinement iterations, producing feasible or near-feasible $(\hat{y}, \hat{f}, \hat{t})$ assignments within milliseconds—orders of magnitude faster than classical MILP solvers.

\section{Neural Surrogate MILP Solver: Intuition}

Classical MILP solvers treat routing as a hard combinatorial problem: variables
must be exactly $0$ or $1$, flows must perfectly satisfy conservation rules, and
capacity and timing constraints must hold with no slack. These constraints carve
out a complicated, high-dimensional polytope that must be searched using
branch-and-bound and repeated LP solves—an approach that becomes extremely slow
for the large, dynamic routing problems encountered in Amazon-scale networks.

Our approach takes a different perspective. Instead of solving the MILP
directly, we train a neural network to \emph{approximate its geometry}. At a
high level, we make three conceptual moves:

\paragraph{1. Replace hard constraints with soft, differentiable surfaces.}
MILP constraints can be viewed as geometric surfaces defining what it means for
a routing plan to be feasible (e.g., ``flow in = flow out'', ``do not exceed lane
capacity''). Instead of enforcing these rules exactly, we convert each
constraint family into a \emph{differentiable penalty} that tells the model how
far it is from feasibility. During training, the network learns to reduce these
penalties through gradient descent, gradually discovering the structure of
feasible routing decisions.

\paragraph{2. Relax discrete decisions into continuous ones.}
Binary lane-activation variables are difficult for gradient-based methods. We
replace them with continuous ``soft'' activations between 0 and 1, allowing the
model to explore fractional choices during learning. As training progresses,
these activations naturally sharpen toward near-binary behavior, effectively
letting the network learn its own approximation of the MILP’s combinatorial
structure. This mirrors the LP relaxation used by classical solvers, but is
learned end-to-end.

\paragraph{3. Use deep networks to imitate solver-style reasoning.}
Instead of constructing solutions in a single step, the model performs a series
of refinement iterations. Each iteration:
\begin{enumerate}[leftmargin=1.3em]
    \item inspects which constraints are violated,
    \item predicts which adjustments would reduce violations,
    \item updates the routing plan using a gradient-like correction.
\end{enumerate}
This behaves like a learned version of LP re-solves, feasibility recovery, and
cutting-plane logic—all compressed into a differentiable neural update.

\paragraph{Constraint Experts.}
MILP constraints fall naturally into families: flow conservation, capacity
limits, binary structure, and timing feasibility. We introduce a set of
\emph{constraint experts}, each specializing in one such family. During
inference, a gating network dynamically selects the most relevant experts,
allowing the model to adapt its reasoning based on the structure of the current
instance. This mirrors how different parts of a classical solver activate
depending on which constraints are tight.

\paragraph{Learning via SGD.}
Because all components of the surrogate are differentiable, the model can be
trained end-to-end using standard stochastic gradient descent (SGD). The loss
encourages:
\begin{itemize}[leftmargin=1.3em]
    \item low transportation cost,
    \item small constraint violations,
    \item near-binary routing decisions,
    \item consistent arrival times and SLA satisfaction.
\end{itemize}
Over many examples, the network learns a smooth approximation of the MILP’s
feasible region—essentially becoming an amortized solver that produces good,
almost-feasible solutions in milliseconds.

\paragraph{Resulting behavior.}
At inference time, the model behaves like a fast, neural version of a MILP
solver:
\begin{itemize}[leftmargin=1.3em]
    \item It predicts which lanes should be active (soft binary decisions).
    \item It allocates flows consistent with capacity.
    \item It adjusts timing and routes to avoid SLA violations.
    \item It refines the entire plan through a few iterations of learned
          ``solver steps''.
\end{itemize}

This substitutes computationally expensive combinatorial search with a learned,
differentiable process that captures the structure of MILP reasoning while
running orders of magnitude faster.

\section{Method}

We propose a \textbf{Neural Surrogate MILP Solver} that approximates mixed-integer 
optimization through three key ideas:
(1) replacing hard MILP constraints with differentiable \emph{soft constraints},
(2) using a \emph{Constraint Mixture-of-Experts (Constraint-MoE)} module to model 
heterogeneous constraint families, and 
(3) applying \emph{latent gradient refinement} to iteratively improve solutions 
using neural gradients instead of MILP branch-and-bound.

Our goal is not to exactly solve MILPs, but to learn an \emph{amortized solver} 
that produces near-feasible, near-optimal solutions in a single forward pass, 
with optional refinement steps. This is particularly valuable in large-scale 
logistics systems such as Amazon middle-mile networks, where thousands of 
MILPs (e.g., VRP, CVRP, pickup-and-delivery, multi-echelon routing) must be 
solved under strict latency budgets.

\subsection{Overview}

Traditional MILPs solve:
\[
\min_{x,z} \; c^\top x \quad \text{s.t. } Ax \le b,\;
x \ge 0,\;
z \in \{0,1\}^m.
\]

Instead, our neural surrogate:
\begin{enumerate}[leftmargin=1.5em]
    \item predicts a relaxed continuous solution $\hat{x}$,
    \item enforces constraints via learned penalty terms,
    \item and projects $\hat{x}$ back toward the feasible polytope using gradient refinement.
\end{enumerate}

This produces a differentiable optimization pipeline that can be trained with SGD.

\subsection{Soft Integer Relaxation}

Binary decision variables $z\in\{0,1\}$ are replaced with 
\[
\hat{z} = \sigma(w), 
\]
where $w$ is a neural logit vector.  
A temperature-controlled sigmoid gives sharper or smoother relaxations:
\[
\hat{z}_i = \sigma(w_i / \tau), \qquad \tau \in (0,1].
\]

As $\tau \rightarrow 0$, $\hat{z}$ approaches discrete values, recovering a MILP-like solution.

\subsection{Differentiable Constraint Penalties}

Each MILP constraint $g_k(x) \le 0$ is converted to a differentiable penalty:
\[
\mathcal{L}_{\text{constraint}} = 
\sum_k \text{ReLU}(g_k(\hat{x}))^2.
\]

Unlike classical Lagrangian relaxation, our penalty weights are \emph{learned}, not hand-tuned.

\subsection{Constraint Mixture-of-Experts (Constraint-MoE)}

Constraints in routing MILPs are heterogeneous:
capacity, flow-balance, assignment, subtour elimination, time windows, etc.

A single penalty network cannot model all of them effectively.

We therefore introduce a \textbf{Constraint MoE}:
\[
\text{Penalty}(g_k) 
= \sum_{e=1}^{E} 
\alpha_{k,e}\,
f_e(g_k(\hat{x})),
\]
where:
\begin{itemize}[leftmargin=1.5em]
    \item $f_e$ is an expert specializing in a constraint family,
    \item $\alpha_{k,e}$ is a gating weight,
    \item the gate learns which expert applies to each constraint.
\end{itemize}

This yields structured learning:
\begin{itemize}
    \item one expert learns capacity violations,
    \item another learns flow conservation,
    \item another learns subtour-like patterns,
    \item another captures combinatorial conflicts.
\end{itemize}

This is crucial for Amazon-style VRP and multi-echelon routing where constraints 
vary dramatically across nodes and customer flows.

\subsection{Neural Feasible Region Approximation}

We approximate the MILP feasible region $\mathcal{F}$ with a neural projector:
\[
\Pi_\theta(\hat{x}) \approx \arg\min_{y \in \mathcal{F}} 
\|y - \hat{x}\|.
\]

Instead of computing exact projections (which require solving another MILP),  
we train a small transformer to produce $y$ using attention over constraint embeddings.

This learns the “shape” of the feasible polytope.

\subsection{Latent Gradient Refinement}

After an initial solution $\hat{x}^{(0)}$ is predicted,  
we perform $T$ refinement steps:
\[
\hat{x}^{(t+1)} 
= \hat{x}^{(t)} 
- \eta \nabla_{\hat{x}} 
\big(
\mathcal{L}_{\text{objective}}
+
\mathcal{L}_{\text{constraint}}
\big).
\]

This refinement:
\begin{itemize}[leftmargin=1.5em]
    \item pushes $\hat{x}$ toward lower cost,
    \item punishes violations more strongly,
    \item mirrors “gradient-based branch-and-bound.”
\end{itemize}

In practice, $T=3$–$5$ gradient steps reduce infeasibility dramatically.

\subsection{Full Surrogate Pipeline}

\begin{enumerate}[leftmargin=1.5em]
    \item Encode MILP instance $(A,b,c)$ using a transformer encoder.
    \item Predict relaxed solution $\hat{x}^{(0)}$ with a decoder.
    \item Apply Constraint-MoE to compute differentiable penalties.
    \item Perform latent gradient refinement.
    \item Optionally discretize $\hat{z}$ with $\tau\rightarrow0$ annealing.
\end{enumerate}

This creates a fully differentiable solver trained end-to-end on optimal (or near-optimal) MILP solutions.

\begin{lstlisting}[style=prettyPython]
import torch
import torch.nn as nn
import torch.nn.functional as F

# ------------------------------------------------------------
# 1. Soft Integer Relaxation Layer
# ------------------------------------------------------------
def relax_binary(logits, tau=0.5):
    """
    Continuous relaxation of z ∈ {0,1}^n.
    Temperature controls hardness.
    """
    return torch.sigmoid(logits / tau)

# ------------------------------------------------------------
# 2. Constraint Experts (MoE)
# ------------------------------------------------------------
class ConstraintExpert(nn.Module):
    def __init__(self):
        super().__init__()
        self.net = nn.Sequential(
            nn.Linear(1, 16),
            nn.ReLU(),
            nn.Linear(16, 1)
        )

    def forward(self, v):
        return self.net(v)

class ConstraintMoE(nn.Module):
    def __init__(self, num_experts=4):
        super().__init__()
        self.experts = nn.ModuleList(
            [ConstraintExpert() for _ in range(num_experts)]
        )
        self.gate = nn.Linear(1, num_experts)

    def forward(self, violation):
        weights = F.softmax(self.gate(violation), dim=-1)
        outputs = []
        for i, expert in enumerate(self.experts):
            outputs.append(weights[:, i:i+1] * expert(violation))
        return torch.stack(outputs, dim=-1).sum(dim=-1)

# ------------------------------------------------------------
# 3. Neural Surrogate MILP Solver
# ------------------------------------------------------------
class NeuralMILPSurrogate(nn.Module):
    def __init__(self, n_vars, num_experts=4):
        super().__init__()
        self.logits = nn.Parameter(torch.randn(n_vars))
        self.moe = ConstraintMoE(num_experts)

    def soft_constraint_penalty(self, x, A, b):
        violation = F.relu(A @ x - b)
        v = violation.unsqueeze(-1)
        penalty = self.moe(v).sum()
        return penalty, violation

    def objective(self, x, c):
        return torch.dot(c, x)

    def forward(self, A, b, c, steps=5, lr=0.1):
        x = relax_binary(self.logits)
        for _ in range(steps):
            obj = self.objective(x, c)
            penalty, _ = self.soft_constraint_penalty(x, A, b)
            loss = obj + penalty
            grad = torch.autograd.grad(loss, x, create_graph=True)[0]
            x = (x - lr * grad).clamp(0, 1)
        return x
\end{lstlisting}

\section{Operational MILPs in Amazon-Scale Logistics}

Large-scale e-commerce networks such as Amazon operate thousands of 
combinatorial optimization problems every day. Most of these problems 
can be expressed as mixed-integer linear programs (MILPs) involving 
binary routing decisions, capacity constraints, batching logic, and 
multi-echelon flow conservation. Below we summarize the core MILPs that 
govern middle-mile and last-mile operations, and show how our Neural 
Surrogate MILP Solver provides differentiable relaxations for each.

\subsection{1. Vehicle Routing Problem (VRP)}

Amazon transportation planners solve VRP-like problems to construct 
multi-stop vehicle tours:
\[
\min_{x_{ij},\,z_i} \;\sum_{i,j} c_{ij} x_{ij}
\]
subject to
\[
\sum_j x_{ij} = z_i, 
\quad 
\sum_i x_{ij} = z_j,
\]
\[
\sum_{i,j} d_i x_{ij} \le \text{Capacity},
\quad
x_{ij} \in \{0,1\}.
\]

\paragraph{Neural relaxation.}
Binary decisions become
\[
\hat{x}_{ij} = \sigma(w_{ij}/\tau),
\]
and subtour and capacity constraints are penalized via
\[
\operatorname{ReLU}(A x - b),
\]
processed by Constraint-MoE experts specializing in
capacity, flow, and subtour patterns.

\subsection{2. Capacitated VRP (CVRP) and Amazon pallet flows}

Middle-mile linehaul planning involves CVRP-like formulations:
\[
\min \sum_{i,j} c_{ij} x_{ij}
\quad \text{subject to}
\]
\[
\sum_{j} x_{ij} = 1,
\quad
\sum_i d_i z_i \le Q,
\quad
z_i \in\{0,1\}.
\]

\paragraph{Neural relaxation.}
Soft capacity violation:
\[
\operatorname{viol}_{\text{cap}}
= \operatorname{ReLU}\Big(\sum_i d_i \hat{z}_i - Q\Big),
\]
with a dedicated MoE capacity expert.

\subsection{Multi-Echelon Middle-Mile Routing}

Amazon’s real network is not a single-hop VRP but a multi-layer 
graph:
\[
\text{FC} \rightarrow \text{SC} \rightarrow \text{Hub} 
\rightarrow \text{DS}.
\]

Flow MILP:
\[
\min \sum_{(i,j)} c_{ij} x_{ij}
\]
subject to
\[
\sum_{j} x_{ij} - \sum_{k} x_{ki} = d_i,
\]
\[
x_{ij} \le \text{Cap}_{ij},
\quad
x_{ij} \ge 0.
\]

\paragraph{Neural relaxation.}
Flow conservation becomes differentiable:
\[
\operatorname{viol}_{\text{flow}} 
= \operatorname{ReLU}\!\left(
\sum_j \hat{x}_{ij} - \sum_k \hat{x}_{ki} - d_i
\right),
\]
with MoE experts capturing flow-balance structure.

\subsection{Shipment Batching and Consolidation}

Amazon must decide how to group outbound parcels into pallets, linehaul 
loads, and containers.

Binary MILP:
\[
y_{ik} = 1 \text{ if shipment } i \text{ assigned to batch } k,
\]
\[
\sum_k y_{ik} = 1,
\quad
\sum_i w_i y_{ik} \le W_k.
\]

\paragraph{Neural relaxation.}
Assignment matrix becomes soft:
\[
\hat{y}_{ik} = \operatorname{Softmax}_k(w_{ik}/\tau),
\]
batch capacity relaxes as:
\[
\operatorname{ReLU}\Big(\sum_i w_i \hat{y}_{ik} - W_k\Big).
\]

\subsection{Hub Selection and Dynamic Consolidation}

Choosing which hubs to activate is a facility-location MILP:
\[
\min \sum_h F_h z_h + \sum_{i,h} c_{ih} x_{ih}
\]
subject to
\[
x_{ih} \le z_h,
\quad
z_h \in \{0,1\}.
\]

\paragraph{Neural relaxation.}
Hub activation becomes:
\[
\hat{z}_h = \sigma(u_h/\tau),
\]
and violations encoded as:
\[
\operatorname{ReLU}(x_{ih} - \hat{z}_h).
\]

\subsection{6. Time Windows and SLA Feasibility}

Each route must satisfy delivery deadlines:
\[
t_j \ge t_i + \text{travel}_{ij} - M(1-x_{ij}),
\]
\[
t_j \le \text{SLA}_j.
\]

\paragraph{Neural relaxation.}
Soft time-window violation:
\[
\operatorname{ReLU}(t_j - \text{SLA}_j),
\]
\[
\operatorname{ReLU}\!\left(
t_i + \text{travel}_{ij} - t_j
\right).
\]
A temporal expert in MoE handles such violations.

\subsection{Neural Surrogate Unification}

All MILPs above can be expressed generically as:
\[
A x \le b, \qquad x \in \{0,1\}^n.
\]

Our surrogate approximates the feasible region using:
\[
x = \sigma(w / \tau),
\quad
\mathcal{L}_{\text{viol}} = 
\sum_k \operatorname{ReLU}(A_k x - b_k),
\]
where each constraint family is processed by a specialized MoE expert.

Refinement via gradient descent:
\[
x^{(t+1)} = \Pi_{[0,1]}\!\big(
x^{(t)} - \eta \nabla_x 
(\mathcal{L}_{\text{obj}} + \mathcal{L}_{\text{viol}})
\big),
\]
yields near-feasible, near-optimal solutions at Amazon scale with 
orders-of-magnitude lower inference cost compared to MILP solvers.

\section{Training}

Training the Neural Surrogate MILP Solver requires converting discrete
optimization problems into a differentiable learning task. Our objective is to
learn a model that produces near-feasible and near-optimal solutions in a single
forward pass, with optional gradient refinement. We train the surrogate using
supervised, self-supervised, and consistency-based losses, depending on the
availability of ground-truth MILP solutions.

\subsection{Training Data}

We generate a large corpus of MILP instances from routing and batching problems
typical of Amazon-scale logistics (VRP, CVRP, multi-echelon routing, pallet
batching, hub selection). For each instance, we store:
\begin{itemize}[leftmargin=1.2em]
    \item the MILP matrices $(A, b, c)$,
    \item optionally an optimal or near-optimal solution $x^\star$ 
          produced by a solver (Gurobi/CPLEX),
    \item metadata such as facility type, lane cost, SLA slack, and capacity
          profiles.
\end{itemize}

When optimal solutions are not available (large-scale instances), we train
purely from constraint satisfaction and self-consistency.

\subsection{Supervised Objective (Optional)}

When a solver's solution $x^\star$ is available, we minimize a prediction loss:
\[
\mathcal{L}_{\text{sup}}
= \| x - x^\star \|_1,
\]
which encourages the surrogate to match the ground-truth MILP decision vector.

To avoid overfitting to solver-specific degeneracies, we include a small entropy
regularizer that keeps the logits smooth:
\[
\mathcal{L}_{\text{ent}} = -\sum_i
\big[ x_i \log x_i + (1-x_i) \log(1-x_i) \big].
\]

\subsection{Constraint Satisfaction Loss}

Regardless of supervision, every MILP instance provides a natural signal via
its constraints:
\[
\mathcal{L}_{\text{viol}}
= \sum_k \operatorname{ReLU}(A_k x - b_k).
\]

Each violation is routed through the \textbf{Constraint-MoE}:
\[
\mathcal{L}_{\text{moe}}
= \sum_k \text{MoE}\big(\operatorname{ReLU}(A_k x - b_k)\big),
\]
where experts specialize in:
capacity, flow conservation, subtours, time windows, assignment consistency,
etc.

MoE gating is trained end-to-end via softmax routing. The sparse expert
activation encourages implicit specialization across constraint families.

\subsection{Objective Approximation Loss}

The surrogate also tries to minimize the original MILP cost:
\[
\mathcal{L}_{\text{obj}}
= c^\top x.
\]

Since $c$ is part of the input, the network must learn how cost structure
interacts with feasibility.

\subsection{Latent Gradient Refinement in Training}

During training we unroll $T$ refinement steps:
\[
x^{(t+1)} = x^{(t)} - \eta \nabla_x 
\big(\mathcal{L}_{\text{obj}} + \mathcal{L}_{\text{moe}}\big),
\]
and backpropagate through all refinement steps (truncated BPTT for stability).

This teaches the model to produce initial predictions $x^{(0)}$ that are already
good and easy to refine, mirroring learned optimization dynamics found in
deep implicit layers.

\subsection{Full Training Objective}

The total training loss is:
\[
\mathcal{L}
=
\lambda_{\text{sup}} \mathcal{L}_{\text{sup}}
+ \lambda_{\text{viol}} \mathcal{L}_{\text{viol}}
+ \lambda_{\text{moe}}  \mathcal{L}_{\text{moe}}
+ \lambda_{\text{obj}}  \mathcal{L}_{\text{obj}}
+ \lambda_{\text{ent}}  \mathcal{L}_{\text{ent}}.
\]

We tune the weights so that:
\begin{itemize}[leftmargin=1.2em]
    \item early training emphasizes feasibility,
    \item later training focuses on cost minimization,
    \item MoE receives increasingly disentangled constraint-specialization signals.
\end{itemize}

\subsection{Generalization Across Network Topologies}

We randomize:
\begin{itemize}[leftmargin=1.2em]
    \item facility counts (FC/SC/Hub/DS),
    \item lane costs and capacities,
    \item demand distributions,
    \item SLA tightness,
    \item congestion patterns.
\end{itemize}

This trains the surrogate to generalize across Amazon’s highly dynamic 
middle-mile network: new demand patterns, outages, disrupted lanes, peak 
season multipliers, and temporary routing changes.

\subsection{Training Efficiency}

Training is performed on GPU/TPU accelerators with batch sizes of 64--256 MILP
instances. Each refinement step is lightweight (one forward+backward pass),
and MoE sparsity ensures compute scales sublinearly with number of experts.

After training, inference is a single forward pass (plus optional 2--3
refinement steps), yielding speedups of 10–100$\times$ compared to MILP
solvers while maintaining competitive feasibility.



%%%%%%%%%%%%%%%%%%%%%%%%%%%%%%%%%%%%%%%%%%%%%%%%%%%%%%%%%%%%
% 4. METHOD
%%%%%%%%%%%%%%%%%%%%%%%%%%%%%%%%%%%%%%%%%%%%%%%%%%%%%%%%%%%%

\section{Method}

\subsection{Architecture Overview}
% Diagram (optional), general flow.

\subsection{Constraint Expert Mixture}
% Show experts specializing:
% - capacity expert
% - flow expert
% - temporal expert
% - feasibility expert

\subsection{Soft Integer Relaxation}
% Sigmoid with temperature
% STE rounding at inference.

\subsection{Latent Gradient Refinement}
% Define energy function
% Gradient update steps.

%%%%%%%%%%%%%%%%%%%%%%%%%%%%%%%%%%%%%%%%%%%%%%%%%%%%%%%%%%%%
% 5. TRAINING
%%%%%%%%%%%%%%%%%%%%%%%%%%%%%%%%%%%%%%%%%%%%%%%%%%%%%%%%%%%%

\section{Training}

% Loss function
% Dataset generation
% Training hyperparameters
% Batch construction

%%%%%%%%%%%%%%%%%%%%%%%%%%%%%%%%%%%%%%%%%%%%%%%%%%%%%%%%%%%%
% 6. EXPERIMENTS
%%%%%%%%%%%%%%%%%%%%%%%%%%%%%%%%%%%%%%%%%%%%%%%%%%%%%%%%%%%%

\section{Experiments}

\subsection{Setup}
% VRP 50/100/200 node

\subsection{Baselines}
% MILP, greedy, neural CO, ablations

\subsection{Results}
% Tables, plots

\subsection{Ablations}
% No experts, no soft relaxation, no refinement

%%%%%%%%%%%%%%%%%%%%%%%%%%%%%%%%%%%%%%%%%%%%%%%%%%%%%%%%%%%%
% 7. ANALYSIS
%%%%%%%%%%%%%%%%%%%%%%%%%%%%%%%%%%%%%%%%%%%%%%%%%%%%%%%%%%%%

\section{Analysis}

% Constraint specialization visualization
% Latent optimization trajectory

%%%%%%%%%%%%%%%%%%%%%%%%%%%%%%%%%%%%%%%%%%%%%%%%%%%%%%%%%%%%
% 8. CONCLUSION
%%%%%%%%%%%%%%%%%%%%%%%%%%%%%%%%%%%%%%%%%%%%%%%%%%%%%%%%%%%%

\section{Conclusion}

% Summary & future work

%%%%%%%%%%%%%%%%%%%%%%%%%%%%%%%%%%%%%%%%%%%%%%%%%%%%%%%%%%%%
% REFERENCES
%%%%%%%%%%%%%%%%%%%%%%%%%%%%%%%%%%%%%%%%%%%%%%%%%%%%%%%%%%%%

\bibliographystyle{plain}
\bibliography{references}

%%%%%%%%%%%%%%%%%%%%%%%%%%%%%%%%%%%%%%%%%%%%%%%%%%%%%%%%%%%%
% APPENDIX
%%%%%%%%%%%%%%%%%%%%%%%%%%%%%%%%%%%%%%%%%%%%%%%%%%%%%%%%%%%%

\appendix

\section{Appendix A: Extended Method}

\section{Appendix B: Additional Experiments}

\end{document}
